\documentclass[12pt,a4paper]{article}
\usepackage[utf8]{inputenc}
\usepackage[english]{babel}
\usepackage[margin=1in]{geometry}

% \usepackage[backend=biber,style=alphabetic,sorting=ynt]{biblatex}
\usepackage[backend=biber,style=numeric,sorting=ynt]{biblatex}
% \usepackage[backend=biber,style=authoryear,sorting=ynt]{biblatex}
% \usepackage[backend=biber,style=authortitle,sorting=ynt]{biblatex}
% \usepackage[backend=biber,style=verbose,sorting=ynt]{biblatex}
% \usepackage[backend=biber,style=reading,sorting=ynt]{biblatex}
% \usepackage[backend=biber,style=draft,sorting=ynt]{biblatex}
\addbibresource{references.bib}

\title{Bibliography management: \texttt{biblatex} package}
\author{Overleaf}
\date{\today}

\begin{document}

\maketitle

\tableofcontents

\section{Demo}
Using \texttt{biblatex} you can display bibliography divided into sections,
depending of citation type.
Let's cite! Einstein's journal paper \cite{einstein} and the Dirac's
book \cite{dirac} are physics related items.
Next, \textit{The \LaTeX\ Companion} book \cite{latexcompanion}, the Donald
Knuth's website \cite{knuthwebsite}, \textit{The Comprehensive Tex Archive
Network} (CTAN) \cite{ctan} are \LaTeX\ related items; but the others Donald
Knuth's items \cite{knuth-fa} are dedicated to programming.

\medskip

% \printbibliography[title={Whole bibliography}]

% \printbibliography[type=article,title={Articles only}]
% \printbibliography[type=book,title={Books only}]
%
% \printbibliography[keyword={physics},title={Physics-related only}]
% \printbibliography[keyword={latex},title={\LaTeX-related only}]

\printbibliography[heading=bibintoc,title={Whole bibliography}]

\end{document}
