\documentclass[12pt,a4paper]{article}
\usepackage[utf8]{inputenc}
\usepackage[utf8]{vietnam}
\usepackage{amsmath}
\usepackage{amsfonts}
\usepackage{amssymb}
\usepackage{indentfirst}
\usepackage[left=2cm,right=2cm,top=2cm,bottom=2cm]{geometry}
%\usepackage[paperheight=8.5cm,left=1cm,right=1cm,top=1cm,bottom=-1cm]{geometry}
\usepackage[unicode,hidelinks=true]{hyperref}
\hypersetup{pdftitle={Ví dụ cách đánh số cho các mục trong lớp article},
	pdfauthor={Thi Minh Nhựt},
	pdfsubject={LaTeX Tutorials},
	pdfkeywords={latex, article, section, subsection, subsubsection, subsubsubsection, paragraph, subparagraph},
	bookmarks=true,
	bookmarksopen=true
}

%% Đánh số đủ 7 cấp
\setcounter{tocdepth}{5}
\setcounter{secnumdepth}{5}

\title{\bfseries \huge Ví dụ cách đánh số cho các mục trong lớp article}
\author{\Large Thi Minh Nhựt \bigskip \\  \Large \texttt{thiminhnhut@gmail.com}}
\date{\Large Ngày 06 tháng 02 năm 2017}

\begin{document}
\maketitle
\tableofcontents
\thispagestyle{empty}
\newpage

\section{Lệnh section}
Nội dung \ldots
\section{Lệnh section}
Nội dung \ldots

\subsection{Lệnh subsection}
Nội dung \ldots
\subsection{Lệnh subsection}
Nội dung \ldots

\subsubsection{Lệnh subsubsection}
Nội dung \ldots
\subsubsection{Lệnh subsubsection}
Nội dung \ldots

\paragraph{Lệnh paragraph}
Nội dung \ldots
\paragraph{Lệnh paragraph}
Nội dung \ldots

\subparagraph{Lệnh subparagraph}
Nội dung \ldots
\subparagraph{Lệnh subparagraph}
Nội dung \ldots

\end{document}