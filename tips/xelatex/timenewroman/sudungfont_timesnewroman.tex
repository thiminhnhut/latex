\documentclass[12pt,a4paper]{article}
\usepackage{fontspec}	% Cho phép sử dụng các font đặc biệt trong TeX với XeLaTeX và LuaLaTeX
\setmainfont{Times New Roman}	% Chọn font Times New Roman làm font chính trong tài liệu

\usepackage{amsmath,amsfonts,amssymb}	% Các font cho phép soạn thảo công thức toán
\usepackage{metalogo}	% Cho phép hiển thị macro \XeLaTeX
\usepackage{indentfirst}	% Thụt vào đầu dòng cho tất cả các đoạn
\usepackage[margin=1in]{geometry}	% Định dạng các lề trong khổ giấy
\usepackage[nodayofweek]{datetime}	% Định dạng cách hiển thị thời gian
\usepackage{booktabs}	% Tạo bảng với gói lệnh booktabs
\usepackage[unicode,hidelinks=true]{hyperref}	% Tạo các siêu liên kết
\hypersetup{pdftitle={Sử dụng font Time New Roman trong LaTeX với XeLaTeX và LuaLaTeX},
  pdfauthor={Thi Minh Nhựt},
  pdfsubject={LaTeX Tutorials},
  pdfkeywords={latex, time new roman latex , xelatex, lualatex},
  bookmarks=true,
  bookmarksopen=true
}


%%% Định nghĩa lại các tên mặc định trong LaTeX
\renewcommand{\contentsname}{Nội dung}
\renewcommand{\refname}{Tài liệu tham khảo}
\renewcommand{\tablename}{Bảng}

%%% Định nghĩa lại các kích thước trong LaTeX
\setlength{\tabcolsep}{25pt}


%%%========================== Tiêu đề của bài viết ===================================%%%
\title{\bfseries \huge Sử dụng font Times New Roman trong \LaTeX\ với \XeLaTeX\ hoặc Lua\LaTeX}
\author{\Large Thi Minh Nhựt \bigskip \\  \Large \texttt{thiminhnhut@gmail.com}}
\date{\Large Ngày 03 tháng 02 năm 2017}
%%%===================================================================================%%%

\begin{document}
\maketitle
\tableofcontents

\begin{thebibliography}{}
  \bibitem{caption-ctan} \href{https://www.ctan.org/author/sommerfeldt}{\textbf{Axel Sommerfeldt}}, \href{https://www.ctan.org/pkg/caption}{\emph{caption – Customising captions in floating environments}}, \href{https://www.ctan.org/}{CTAN -- Comprehensive TEX Archive Network}, \formatdate{22}{05}{2016}.

  \bibitem{fontspec-chandrahas} \href{https://goo.gl/quEV70}{\textbf{Chandra Has}}, \href{https://goo.gl/ljfpxm}{\emph{LaTeX: Complete basic tutorial by Chandra Has}}, \href{https://goo.gl/xiJ8hm}{\emph{Lession 35 -- Use External Fonts With "Fontspec" package (Latex Tutorial)}}, \href{https://www.youtube.com/}{Youtube}, \formatdate{10}{5}{2014}.

  \bibitem{caption-table-stackexchange} Aks: \href{https://goo.gl/Iv3GDx}{\textbf{Ignacio Gómez}} -- Answered: \href{https://goo.gl/I4Lbvw}{\textbf{azetina}}, \href{https://goo.gl/2wcY14}{\emph{Change the word ``Table'' in table captions}}, \href{http://tex.stackexchange.com/}{TeX - LaTeX Stack Exchange}, \formatdate{05}{11}{2012}.

  \bibitem{fontspec-stackexchange} Aks: \href{https://goo.gl/oHHZ2J}{\textbf{Karel Bílek}} -- Answered: \href{https://goo.gl/Wk6MGR}{\textbf{egreg}}, \href{https://goo.gl/JMexOO}{\textbf{cfr}}, \href{https://goo.gl/ITcJ7j}{\emph{How to easily use UTF-8 with LaTeX?}}, \href{http://tex.stackexchange.com/}{TeX - LaTeX Stack Exchange}, \formatdate{19}{04}{2014}.

  \bibitem{nguyenhuudien} \textbf{Nguyễn Hữu Điền -- Nguyễn Minh Tuấn}, \emph{\LaTeX: Tra cứu và soạn thảo -- Chương 6: Kỹ thuật tự tạo lệnh}, Trang 145, NXB ĐHQG Hà Nội, Năm 2001.

  \bibitem{times-ctan} \href{https://www.ctan.org/author/rahtz}{\textbf{Se­bas­tian Rahtz}}, \href{https://www.ctan.org/pkg/times}{\emph{times – Select Adobe Times Roman (or equivalent) as default font}}, \href{https://www.ctan.org/}{CTAN -- Comprehensive TEX Archive Network}.

  \bibitem{fontspec-ctan} \href{https://www.ctan.org/author/robertson}{\textbf{Will Robertson}}, \href{https://www.ctan.org/pkg/fontspec}{\emph{fontspec – Advanced font selection in \XeLaTeX\ and Lua\LaTeX}}, \href{https://www.ctan.org/}{CTAN -- Comprehensive TEX Archive Network}, \formatdate{01}{02}{2016}.
\end{thebibliography}

\section{Giới thiệu}
Khi viết các báo cáo tiểu luận và luận văn nhiều trường Đại học ở Việt Nam có các quy định về hình thức trình bày báo cáo, cụ thể là vấn đề sử dụng font Times New Roman trong bài báo cáo.\\

Tài liệu được soạn thảo bằng \LaTeX\ có font mặc định không phải là font Times New Roman. Nên vấn đề trên được giải quyết với gói lệnh \verb|fontspec|~\cite{fontspec-ctan} khi biên dịch với \XeLaTeX\ hoặc Lua\LaTeX, giúp cho người sử dụng \LaTeX\ có thể viết báo cáo với font Time New Roman theo yêu cầu.\\

Ngoài font Time New Roman, gói lệnh \verb|fontspec| cũng cho  phép sử dụng các font đặc biệt khác (như font Arial, Courier New, Ubuntu Mono,\ldots) trong tài liệu được soạn thảo bằng \LaTeX\ khi biên dịch với \XeLaTeX\ hoặc Lua\LaTeX. Khi biên dịch bằng \XeLaTeX\ thì có thể sử dụng trực tiếp các font chữ được cài đặt sẵn trên máy tính (có thể sử dụng được các font chữ bên Microsoft Word).\\

Ngoài ra, \LaTeX\ cũng cung cấp gói lệnh \verb|times|~\cite{times-ctan}, tạo được font chữ gần giống với font Times New Roman (khoảng $99\%$).\\

Phần hướng dẫn bên dưới đã được thử nghiệm thành công với phiên bản \TeX Live 2015 được cài đặt trên hệ điều hành Ubuntu 16.04 và sử dụng trình soạn thảo \TeX Maker để biên dịch với \XeLaTeX. \\

File \TeX\ của bài hướng dẫn được lưu ở địa chỉ \url{https://github.com/thiminhnhut/latex/tree/master/tips/xelatex/timenewroman}, chúng ta có thể dùng file này để làm mẫu thực hiện soạn theo.

\section{Cách khai báo sử dụng font Times New Roman trong \LaTeX}
\subsection{Sử dụng font Times New Roman với gói lệnh fontspec trong \XeLaTeX\ hoặc Lua\LaTeX}
Để sử dụng font Times New Roman, ta cần khai báo các lệnh và chú ý các điều sau:
\begin{itemize}
  \item Phần khai báo trước \verb|\begin{document}| cần có hai dòng bên dưới~\cite{fontspec-chandrahas, fontspec-stackexchange}:
        \begin{verbatim}
          \usepackage{fontspec}
          \setmainfont{Times New Roman}
        \end{verbatim}

  \item Lưu ý:
        \begin{itemize}
          \item Để hiển thị tiếng Việt không bị lỗi font trong tài liệu (trong file pdf) khi biên dịch với \XeLaTeX\ hoặc Lua\LaTeX\ không cần phải khai báo \verb|\usepackage[utf8]{inputenc}| và \verb|\usepackage[utf8]{vietnam}|.

          \item Đảm bảo máy tính của bạn đã cài đặt font Times New Roman.

          \item Chọn XeLaTeX hoặc LuaLaTeX để biên dịch. Phần soạn thảo thực hiện như bình thường.
        \end{itemize}

  \item Cách này gặp hạn chế là cần phải định nghĩa lại các tên mặc định trong \LaTeX\ để cho hiển thị tiếng Việt. Vì các tên này đều hiển thị bằng tiếng Anh, ví dụ: \verb|\tableofcontents| cho hiển thị là Contents, \verb|\listoffigures| cho hiển thị là List of Figures,\ldots
        \begin{itemize}
          \item Trên bảng~\ref{Tab:recommand-name} trình bày tên trong tiếng Anh và tên trong tiếng Việt ứng với các lệnh trong \LaTeX~\cite{nguyenhuudien}. Để định nghĩa lại với tên mới chúng ta dùng lệnh:
                \begin{verbatim}
              \renewcommand{Lệnh}{Tên mới}
            \end{verbatim}

                \begin{table}[ht]
                  \begin{center}
                    \begin{tabular}{lll}
                      \toprule
                      \textbf{Lệnh}              & \textbf{Tên tiếng Anh} & \textbf{Tên tiếng Việt} \\ \midrule
                      \verb|\abstractname|       & Abstract               & Tóm tắt                 \\ \midrule
                      \verb|\alsoname|           & Also                   & Cũng vậy                \\ \midrule
                      \verb|\alsoseename|        & Also see               & Cũng vậy xem            \\ \midrule
                      \verb|\appendixname|       & Appendix               & Phụ lục                 \\ \midrule
                      \verb|\bibname|            & Bibliography           & Tài liệu                \\ \midrule
                      \verb|\ccname|             & Cc                     & Cc                      \\ \midrule
                      \verb|\chaptername|        & Chapter                & Chương                  \\ \midrule
                      \verb|\contentsname|       & Contents               & Mục lục                 \\ \midrule
                      \verb|\datename|           & Date                   & Ngày                    \\ \midrule
                      \verb|\enclname|           & Enclosure              & Kèm theo                \\ \midrule
                      \verb|\figurename|         & Figure                 & Hình                    \\ \midrule
                      \verb|\indexname|          & Index                  & Chỉ số                  \\ \midrule
                      \verb|\keywordsname|       & Key words              & Khóa từ                 \\ \midrule
                      \verb|\listfigurename|     & List of Figures        & Danh sách các hình      \\ \midrule
                      \verb|\listtablename|      & List of Tables         & Danh sách các bảng      \\ \midrule
                      \verb|\lstlistlistingname| & Listings               & Danh sách chương trình  \\ \midrule
                      \verb|\lstlistingname|     & Listing                & Chương trình            \\ \midrule
                      \verb|\notesname|          & Notes                  & Ghi chú                 \\ \midrule
                      \verb|\headpagename|       & Page                   & Trang                   \\ \midrule
                      \verb|\pagename|           & Page                   & Trang                   \\ \midrule
                      \verb|\partname|           & Part                   & Phần                    \\ \midrule
                      \verb|\proofname|          & Proof                  & Chứng minh              \\ \midrule
                      \verb|\refname|            & References             & Tài liệu tham khảo      \\ \midrule
                      \verb|\tablename|          & Table                  & Bảng                    \\ \midrule
                      \verb|\preffacename|       & Preface                & Lời nói đầu             \\ \midrule
                      \verb|\seename|            & See                    & Xem                     \\ \midrule
                      \verb|\subjectname|        & Subject                & Chủ đề                  \\ \midrule
                      \verb|\tocname|            & Table of Contents      & Bảng danh mục           \\ \midrule
                      \verb|\contentsname|       & Table of Contents      & Bảng danh mục           \\ \midrule
                      \verb|\headtoname|         & To                     & Đến                     \\
                      \bottomrule
                    \end{tabular}
                  \end{center}
                  \caption{Danh sách tên ứng với các lệnh trong \LaTeX}\label{Tab:recommand-name}
                \end{table}

          \item Ngoài cách trên, gói lệnh \verb|caption|~\cite{caption-ctan} với các tùy chọn cho phép thay đổi linh hoạt hơn, khai báo như sau~\cite{caption-table-stackexchange}:
                \begin{verbatim}
              \usepackage{caption}
              \captionsetup[table]{name=Tên mới}
              \captionsetup[figure]{name=Tên mới}
            \end{verbatim}
        \end{itemize}
\end{itemize}

\subsection{Sử dụng font Times New Roman với gói lệnh times}
Với gói lệnh \verb|times|, cho font chữ giống font Time New Roman khoảng $99\%$.
\begin{itemize}
  \item Phần khai báo trước \verb|\begin{document}| cần có hai dòng bên dưới:
        \begin{verbatim}
          \usepackage[utf8]{inputenc}
          \usepackage[utf8]{vietnam}
          \usepackage{times}
        \end{verbatim}

  \item Chọn PDFLaTeX để biên dịch. Phần soạn thảo thực hiện như bình thường, các tên mặc định được hiển thị bằng tiếng Việt.
\end{itemize}
\end{document}
