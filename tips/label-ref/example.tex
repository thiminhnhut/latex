\documentclass[12pt,a4paper]{article}
\usepackage[utf8]{vietnam}
\usepackage{amsmath}
\usepackage{amsfonts}
\usepackage{amssymb}
\usepackage{graphicx}
\usepackage{multicol}
\usepackage{indentfirst}
\usepackage[unicode,hidelinks=true]{hyperref}
\usepackage[left=2cm,right=2cm,top=2cm,bottom=2cm]{geometry}
\everymath{\displaystyle}


\title{Ví dụ về cách tham chiếu công thức toán học, tên hình, tên bảng trong \LaTeX}
\author{Thực hiện: Thi Minh Nhựt (Email: \texttt{thiminhnhut@gmail.com})}
\date{Thời gian: \today}

\begin{document}

\maketitle

\section{Tham chiếu công thức toán học}
\begin{itemize}
  \item Sử dụng các môi trường \verb|equation, align,...| để viết công thức toán học, rồi đặt lệnh \verb|\label{Eq:Name Label}| vào trong mỗi môi trường. Để tham chiếu lại công thức thì dùng lệnh \verb|\ref{Eq:Name Label}|.
  \item Code mẫu:
        \begin{verbatim}
          \begin{equation} \label{Eq:Pytago}
            a^2 + b^2 = c^2
          \end{equation}

          Định lý Pytago được biểu diễn bằng công thức (\ref{Eq:Pytago})
        \end{verbatim}
  \item Kết quả cho đoạn code ở trên:
        \begin{equation} \label{Eq:Pytago}
          a^2 + b^2 = c^2
        \end{equation}

        Định lý Pytago được biểu diễn bằng công thức (\ref{Eq:Pytago})
\end{itemize}

\section{Tham chiếu hình ảnh}
\begin{itemize}
  \item Sử dụng các môi trường \verb|figure| để chèn hình ảnh, rồi đặt lệnh \verb|\label{Fig:Name Label}| vào trong mỗi môi trường. Để tham chiếu lại hình ảnh thì dùng lệnh \verb|\ref{Fig:Name Label}|.
  \item Code mẫu:
        \begin{verbatim}
          \begin{figure}[htp]
            \begin{center}
              % Chèn hình vào đây
            \end{center}
            \caption{Hình minh họa} \label{Fig:Hinh-minh-hoa}
          \end{figure}

          Hình ảnh minh họa (Hình \ref{Fig:Hinh-minh-hoa}).
        \end{verbatim}
  \item Kết quả cho đoạn code ở trên:

        \begin{figure}[!htp]
          \begin{center}
            \includegraphics[scale=0.5]{images/pytago}
          \end{center}
          \caption{Hình minh họa} \label{Fig:Hinh-minh-hoa}
        \end{figure}

        Hình ảnh minh họa (Hình \ref{Fig:Hinh-minh-hoa}).
\end{itemize}

\section{Tham chiếu bảng}
\begin{itemize}
  \item Sử dụng các môi trường \verb|table| để viết công chèn bảng, rồi đặt lệnh \verb|\label{Tab:Name Label}| vào trong mỗi môi trường. Để tham chiếu lại bảng thì dùng lệnh \verb|\ref{Tab:Name Label}|.
  \item Code mẫu:
        \begin{verbatim}
          \begin{table}[htp]
            \begin{center}
              % Chèn bảng vào đây
            \end{center}
            \caption{Bảng minh họa} \label{Tab:Bang-minh-hoa}
          \end{table}

          Bảng minh họa (Bảng \ref{Tab:Bang-minh-hoa}).
        \end{verbatim}
  \item Kết quả cho đoạn code ở trên:

        \begin{table}[!htp]
          \caption{Bảng minh họa} \label{Tab:Bang-minh-hoa}
          \begin{center}
            {
              \renewcommand{\arraystretch}{1.3}
              \begin{tabular}{|c|c|c|c|c|} \hline
                \(a\) & \(b\) & \(c\) & \(a^2 + b^2\)      & \(c^2\)      \\ \hline
                3     & 4     & 5     & \(3^2 + 4^2 = 25\) & \(5^2 = 25\) \\ \hline
              \end{tabular}
            }
          \end{center}
        \end{table}

        Bảng minh họa (Bảng \ref{Tab:Bang-minh-hoa}).
\end{itemize}
\end{document}
